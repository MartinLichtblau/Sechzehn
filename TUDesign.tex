% accentcolor - https://www.tu-darmstadt.de/media/medien_stabsstelle_km/services/medien_cd/das_bild_der_tu_darmstadt.pdf (pages 17-18)
% amsmath	- formatting math stuff	(ftp://ftp.ams.org/pub/tex/doc/amsmath/amsldoc.pdf)
% enumitem	- formatting enumerates	(http://ftp.uni-erlangen.de/ctan/macros/latex/contrib/enumitem/enumitem.pdf)
% graphicx	- images				(http://ctan.space-pro.be/tex-archive/macros/latex/required/graphics/grfguide.pdf)
% lipsum	- lorem ipsum			(ftp://ftp.fu-berlin.de/tex/CTAN/macros/latex/contrib/lipsum/lipsum.pdf)
% listings	- source code			(http://texdoc.net/texmf-dist/doc/latex/listings/listings.pdf)
% mathtools	- amsmath extension		(ftp://ftp.fu-berlin.de/tex/CTAN/macros/latex/contrib/mathtools/mathtools.pdf)
% parskip	- splittitle 
\documentclass[11pt, accentcolor=tud1c]{tudreport}	
\usepackage[english]{babel}
\usepackage[utf8]{inputenc}
\usepackage{amsmath, enumitem, graphicx, hyperref, lipsum, listings, mathtools, parcolumns}

% https://scm.informatik.tu-darmstadt.de/news/129-technical-documentation-sample


% Specifying new commands
\newcommand{\titlerow}[2]{
    \begin{parcolumns}[colwidths={1=.15\linewidth}]{2}
        \colchunk[1]{#1:}
        \colchunk[2]{#2}
    \end{parcolumns}
    \vspace{0.2cm}
}

% configuring title
\title{Internet-Praktikum: Telekooperation}
\setinstitutionlogo{./res/logo.pdf}

\subtitle{
	\titlerow{Project}{Sechzehn}
	\titlerow{Team Bravo}{Alexander Geiß {\normalsize(alexanderhelmut.geiss@stud.tu-darmstadt.de)}, \\ 
	                      Lukas Klein {\normalsize(lukas.klein@stud.tu-darmstadt.de)},  \\ 
	                      Martin Lichtblau {\normalsize(martin.lichtblau@stud.tu-darmstadt.de)}, \\ 
	                      Johannes Semsch {\normalsize(johannesmaximilianchristian.semsch@stud.tu-darmstadt.de)}, \\ 
	                      Tim Walter {\normalsize(tim.walter.10@stud.tu-darmstadt.de)}}
}



% document 
\begin{document}

\maketitle
\tableofcontents

\chapter{Motivation}\label{ch:motivation}
% ! max. 1 page !


\chapter{Overview}\label{ch:overview}
% ! image demanded !
In this section we present the architecture we used and thereby how our components interact with each other. We then briefly describe which technologies we chose, and why we chose them.
\section{Architecture}\label{sec:architecture}
% components and their interaction
\subsection{Diagram Interaction}
\subsection{ERM}
\subsection{UML Diagram}
\section{Server}\label{sec:server}
To host our server we use the Heroku Cloud Application Platform\cite{heroku}. Heroku provides us a PostgreSQL\cite{postgresql} database service and an nodejs execution environment. Furthermore it offers access to Cloudinary\cite{cloudinary} image backend. The Cloudinary service can be used as plugin within heroku.
\subsection{Node.js}
\cite{nodejs}
\subsection{AdonisJs}\label{sec:adonisjs}
\cite{adonisjs}
\subsection{PostgreSQL}
PostgreSQL\cite{postgresql} is an open-source relational database management system. It is possible to add new data types and functions in PostgreSQL. Amongst other things we implemented a function to compute the Great-circle distance\cite{greatcircledist}. The Great-circle distance is necessary to compute distances on spheres and therefore to compute distances on the earth. Moreover native programming interfaces exist for many languages. Since Heroku also has a PostgreSQL plugin, that was our way to go.
\subsection{JSON Web Tokens}\label{sec:json_web_tokens}
The Authentication in our project is done with JSON Web Tokens(JWTs)\cite{jwt}.

\section{Client}\label{sec:client}
\subsection{Android Data Binding}
\cite{databinding}
\subsection{Retrofit}
\cite{retrofit}
\subsection{GSON}
\cite{gson}
\subsection{ChatKit}
\cite{chatkit}

% TODO
% technologies used
% interfaces
% pls explain your design decisions!
% pls explain the particular components, interfaces,..(e.g., app, database, API,..) in detail.


\chapter{Feature Explanation}\label{ch:feature_explanation}


\section{System Requirements}
minSDK 23 \\
compileSDK 25 \\
Permissions and justification why they are necessary
\section{Get Started}
Explain how to download and install the APK
\section{App Flow}
% system requirements, get started, functionalities, features, app flow, ...
When a user starts the app for the first time he enters the login screen. On this screen the user has three options:
\begin{enumerate}
\item he can login with his email and password.
\item if he has no account yet, he can enter the register screen and create one.
\item if he has forgotten his password he can request a password reset link to your email address.
\end{enumerate}
We further assume that the user is logged in. he now sees a bottom naigation bar with three navigation options.
\begin{enumerate}
\item the first one shows the map with venues and users
\item the second one shows a list with friends and messages
\item the third one is the users profile and its settings
\end{enumerate}
% Diagramm einfuegen?

\chapter{Feature List}\label{ch:feature_list}
% ! 1 page !
% pls only list the most relevant implemented features
Bonus Feature: Email Verficiation, JWT, Filterable Results, Live Update, Anonymity Button
% Bibliography
\begin{thebibliography}{9}
	\bibitem{heroku}
		\textit{Heroku}, 
		Accessed: 2017-08-09, \\
		\url{https://www.heroku.com/}
	\bibitem{cloudinary}
		\textit{Cloudinary Features}, 
		Accessed: 2017-08-09 \\
		\url{http://cloudinary.com/features}
	\bibitem{nodejs}
		\textit{API Reference Documentation}, 
		Accessed: 2017-08-09, \\
		\url{https://nodejs.org/en/docs/}
	\bibitem{adonisjs}
		\textit{AdonisJs at a Glance}, 
		Accessed: 2017-08-09, \\
		\url{https://adonisjs.com/docs/3.2/overview}
	\bibitem{postgresql}
		\textit{PostgreSQL: Documentation}, 
		Accessed: 2017-08-09, \\
		\url{https://www.postgresql.org/docs/}
	\bibitem{greatcircledist}
		\textit{Great-Circle Distanz}, 
		Accessed: 2017-08-09, \\
		\url{https://en.wikipedia.org/wiki/Great-circle_distance}
	\bibitem{jwt}
		\textit{Introduction to JSON Web Tokens}, 
		Accessed: 2017-08-09, \\
		\url{https://jwt.io/introduction/}
	\bibitem{databinding}
		\textit{Data Binding Library}, 
		Accessed: 2017-08-09, \\
		\url{https://developer.android.com/topic/libraries/data-binding/index.html}
	\bibitem{retrofit}
		\textit{Retrofit: A type-safe HTTP client for Android and Java}, 
		Accessed: 2017-08-09, \\	
		\url{https://square.github.io/retrofit/}
	\bibitem{gson}
		\textit{Google GSON}, 
		Accessed: 2017-08-09, \\
		\url{https://github.com/google/gson}
	\bibitem{chatkit}
		\textit{ChatKit for Android}, 
		Accessed: 2017-08-09, \\
		\url{https://github.com/stfalcon-studio/ChatKit}
\end{thebibliography}
\end{document}