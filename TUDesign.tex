% accentcolor - https://www.tu-darmstadt.de/media/medien_stabsstelle_km/services/medien_cd/das_bild_der_tu_darmstadt.pdf (pages 17-18)
% amsmath	- formatting math stuff	(ftp://ftp.ams.org/pub/tex/doc/amsmath/amsldoc.pdf)
% enumitem	- formatting enumerates	(http://ftp.uni-erlangen.de/ctan/macros/latex/contrib/enumitem/enumitem.pdf)
% graphicx	- images				(http://ctan.space-pro.be/tex-archive/macros/latex/required/graphics/grfguide.pdf)
% lipsum	- lorem ipsum			(ftp://ftp.fu-berlin.de/tex/CTAN/macros/latex/contrib/lipsum/lipsum.pdf)
% listings	- source code			(http://texdoc.net/texmf-dist/doc/latex/listings/listings.pdf)
% mathtools	- amsmath extension		(ftp://ftp.fu-berlin.de/tex/CTAN/macros/latex/contrib/mathtools/mathtools.pdf)
% parskip	- splittitle 
\documentclass[accentcolor=tud1c]{tudreport}	
\usepackage[english]{babel}
\usepackage[utf8]{inputenc}
\usepackage{amsmath, enumitem, graphicx, hyperref, lipsum, listings, mathtools, parcolumns}

% https://scm.informatik.tu-darmstadt.de/news/129-technical-documentation-sample


% Specifying new commands
\newcommand{\titlerow}[2]{
    \begin{parcolumns}[colwidths={1=.15\linewidth}]{2}
        \colchunk[1]{#1:}
        \colchunk[2]{#2}
    \end{parcolumns}
    \vspace{0.2cm}
}

% configuring title
\title{Internet-Praktikum: Telekooperation}
\setinstitutionlogo{./res/logo.pdf}

\subtitle{
	\titlerow{Project}{Sechzehn}
	\titlerow{Team Bravo}{Alexander Geiß {\normalsize(alexanderhelmut.geiss@stud.tu-darmstadt.de)}, \\ 
	                      Lukas Klein {\normalsize(lukas.klein@stud.tu-darmstadt.de)},  \\ 
	                      Martin Lichtblau {\normalsize(martin.lichtblau@stud.tu-darmstadt.de)}, \\ 
	                      Johannes Semsch {\normalsize(johannesmaximilianchristian.semsch@stud.tu-darmstadt.de)}, \\ 
	                      Tim Walter {\normalsize(tim.walter.10@stud.tu-darmstadt.de)}}
}



% document 
\begin{document}

\maketitle
\tableofcontents

\chapter{Motivation}\label{ch:motivation}
% ! max. 1 page !


\chapter{Overview}\label{ch:overview}
% ! image demanded !
\section{Architecture}\label{sec:architecture}
In the following sections we describe the technologies used by us.
% components and their interaction
\section{Server}\label{sec:server}
Server is hosted on Heroku\cite{heroku}
\subsection{Node.js}
\cite{nodejs}
\subsection{AdonisJs}\label{sec:adonisjs}
\cite{adonisjs}
\subsection{PostgreSQL}
\cite{postgresql}
\subsection{JSON Web Tokens}\label{sec:json_web_tokens}
The Authentication in our project is done with JSON Web Tokens(JWTs)\cite{jwt}.

\section{Client}\label{sec:client}
\subsection{Android Data Binding}
\cite{databinding}
\subsection{Retrofit}
\cite{retrofit}
\subsection{GSON}
\cite{gson}
\subsection{ChatKit}
\cite{chatkit}

% TODO
% technologies used
% interfaces
% pls explain your design decisions!
% pls explain the particular components, interfaces,..(e.g., app, database, API,..) in detail.


\chapter{Feature Explanation}\label{ch:feature_explanation}
\section{System Requirements}
\section{Get Started}
\section{App Flow}
% system requirements, get started, functionalities, features, app flow, ...

\chapter{Feature List}\label{ch:feature_list}
% ! 1 page !
% pls only list the most relevant implemented features

% Bibliography
\begin{thebibliography}{9}
	\bibitem{heroku}
		\textit{Heroku}, 
		Accessed: 2017-08-09, \\
		\url{https://www.heroku.com/}
	\bibitem{nodejs}
		\textit{API Reference Documentation}, 
		Accessed: 2017-08-09, \\
		\url{https://nodejs.org/en/docs/}
	\bibitem{adonisjs}
		\textit{AdonisJs at a Glance}, 
		Accessed: 2017-08-09, \\
		\url{https://adonisjs.com/docs/3.2/overview}
	\bibitem{postgresql}
		\textit{PostgreSQL: Documentation}, 
		Accessed: 2017-08-09, \\
		\url{https://www.postgresql.org/docs/}
	\bibitem{jwt}
		\textit{Introduction to JSON Web Tokens}, 
		Accessed: 2017-08-09, \\
		\url{https://jwt.io/introduction/}
	\bibitem{databinding}
		\textit{Data Binding Library}, 
		Accessed: 2017-08-09, \\
		\url{https://developer.android.com/topic/libraries/data-binding/index.html}
	\bibitem{retrofit}
		\textit{Retrofit: A type-safe HTTP client for Android and Java}, 
		Accessed: 2017-08-09, \\	
		\url{https://square.github.io/retrofit/}
	\bibitem{gson}
		\textit{Google GSON}, 
		Accessed: 2017-08-09, \\
		\url{https://github.com/google/gson}
	\bibitem{chatkit}
		\textit{ChatKit for Android}, 
		Accessed: 2017-08-09, \\
		\url{https://github.com/stfalcon-studio/ChatKit}
\end{thebibliography}
\end{document}